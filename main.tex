%----------------------------------------------------------------------------------------
%	PACKAGES AND OTHER DOCUMENT CONFIGURATIONS
%----------------------------------------------------------------------------------------

\documentclass[9pt]{developercv}
\usepackage{multicol}
\setlength{\columnsep}{0mm}
%----------------------------------------------------------------------------------------
\usepackage{lipsum}  


\begin{document}

%----------------------------------------------------------------------------------------
%	TITLE AND CONTACT INFORMATION
%----------------------------------------------------------------------------------------

\begin{minipage}[t]{0.5\textwidth} 
	\vspace{-\baselineskip}
	{ \fontsize{16}{20} \textcolor{black}{\textbf{\MakeUppercase{Гордеев Александр}}}}
	
	\vspace{6pt}
	
	{\Large Fullstack Developer}
\end{minipage}
\hfill
\begin{minipage}[t]{0.2\textwidth}
	\vspace{-\baselineskip}
    \faIcon{phone}{ +7 }{(920) 338-95-98}\\\\
    \faIcon{telegram}{\href{https://t.me/almanelis}{ t.me/almanelis}}\\\\
    \faIcon{map-marker}{ Москва, Россия}\\
	
\end{minipage}
\begin{minipage}[t]{0.27\textwidth}
	\vspace{-\baselineskip}
	\faIcon{envelope}{\href{mailto:alexgordeev231@gmail.com}{ alexgordeev231@gmail.com}}\\\\
    \faIcon{github}{\href{https://github.com/almanelis}{ github.com/almanelis}}\\
       
    
\end{minipage}


%----------------------------------------------------------------------------------------
%	INTRODUCTION, SKILLS AND TECHNOLOGIES
%----------------------------------------------------------------------------------------

\begin{minipage}[t]{0.46\textwidth}
    \cvsect{Обо мне}
	\vspace{-6pt}
 
        Fullstack разработчик в медицине, студент Сеченовского Университета по направлению обучения Информационные
        системы и технологии. Занимаюсь разработкой сервисов в медицине, анализом данных, интеграцией нейросетевых
        моделей.
        \\
\end{minipage}
\hfill % Whitespace between
\begin{minipage}[t]{0.465\textwidth}
    \cvsect{Область специализации}
    \vspace{-6pt}
    
    Web - Backend - Frontend - Data Analyse - \\
    Search engines - NER - CNN - Medical Image Processing - Open Source software
    
\end{minipage}

%----------------------------------------------------------------------------------------
%	Projects
%----------------------------------------------------------------------------------------
\vspace{-10 pt}
\cvsect{Профессиональный опыт}
\begin{entrylist}
    \entry
		{Разработчик}
		{ООО "РЛС-Патент"}
		{04/2023 - н.в.}
		{
            \begin{itemize}[itemsep=0.5cm, nolistsep, leftmargin=-2cm]
                \item Коммерческая разработка сервисов РЛС для последующей интеграцией в МО.
                \item Разработка сервиса "'РЛС-Противопоказания"' основанного на анализе противопоказаний из инструкций лекарственных препаратов.
                \item Разработка сервиса "'РЛС-Гармонизация"' для автоматизации сопоставления лекарственных номенклатур.
                \item Автоматизация разметки с помощью Pandas и регулярных выражений, для последующего использования NLTK, Natasha.
                \item Обучение лингвистической модели на основе BERT для задачи NER.
                \item Протототипирование сервисов. Разработка API на Django, нативный фронт с использованием CSS фреймворков. Адаптирование и предоставление данных из PostgreSQL. 
            \end{itemize}
            }
\end{entrylist}

%----------------------------------------------------------------------------------------
%	EDUCATION
%----------------------------------------------------------------------------------------
\vspace{-15 pt}
\cvsect{Высшее образование}
\begin{entrylist}
    \entry
		{09/2021 -- н.в.}
		{Сеченовский Университет}
		{Бакалавриат}
		{09.03.02 Информационные системы и технологии}
\end{entrylist}

%----------------------------------------------------------------------------------------
%	EXPERIENCE
%----------------------------------------------------------------------------------------
\vspace{-10 pt}
\cvsect{Курсы}
\begin{entrylist}
	\entry
        {09/2023 -- 05/2024}
		{DevOps в медицине}
		{Ростелеком}
		{\vspace{-10pt} 
        \texttt{DevOps инженер}}
	\entry
		{09/2023 -- 05/2024}
		{Разработчик цифровых медицинских сервисов}
		{Яндекс Практикум}
		{\vspace{-10pt}
        \texttt{Backend-разработчик} \slashsep \texttt{Data-science} \slashsep \texttt{Машинное обучение}}
\end{entrylist}

\vspace{-10 pt}
\cvsect{Конкурсы}
\begin{entrylist}
	\entry
        {08/2023}
		{Хакатон INNOGLOBALHACK}
		{Иннополис}
		{\vspace{-10pt} \\
                    \begin{itemize}[nolistsep, leftmargin=-2cm]
                \item Интеграция API сервиса для организации спринтов в Telegram.
                \item Разработка телеграм бота на aiogram для работы с CRUD методами API сервиса. 
            \end{itemize}}
        \entry
        {10/2023}
        {Инженерный хакатон}
        {Сеченовский Университет}
        {\vspace{-10pt} \\
                    \begin{itemize}[nolistsep, leftmargin=-2cm]
                \item Разработка мобильного приложения с интегрированием CNN модели для распознования и классификации эмоций людей
с расстройством аутистического спектра.

                \item Организация разметки изображений с помощью DropBox.
                \item Обучение CNN модели с помощью Keras, TensorFlow, тестирование на OpenCV.

                \item Конвертация модели в TensorFlowLight и её интеграция в локальную архитектуру мобильного приложения на Flutter.

            \end{itemize}}
\end{entrylist}

\vspace{-10 pt}
\cvsect{Навыки}
\textbf{Языки программирования:} {Python(основной), JavaScript, C++, R
}\\\\
\textbf{Библиотеки и фреймворки:} {: Django, FastAPI, SqlAlchemy, alembic, Celery, Pandas, Numpy, TensorFlow, Keras,
OpenCV, NLTK, Natasha, Aoigram, jQuery, TailwindCSS}\\\\
\textbf{Software:} { Git, Docker, Kubernetes, ElasticSearch, Microsoft Office}\\\\
\textbf{OS:} {Linux, Windows
}\\\\
\textbf{Soft Skills:} {Аналитическое мышление, Планирование, Организация, Креативное решение проблем, Командная
работа}\\

%----------------------------------------------------------------------------------------
%	LANGUAGES
%----------------------------------------------------------------------------------------
\vspace{-10 pt}
	\cvsect{Языки}
    \vspace{-6pt}
    
    \hspace{26mm} \textbf{English} - B2, \textbf{ Русский} - родной

%----------------------------------------------------------------------------------------

\end{document}
