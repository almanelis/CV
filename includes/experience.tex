\section{Профессиональный опыт}

\customcventry{04/2023 ‐ н.в.}{{(ООО "РЛС-Патент")}}{IT-специалист}{Москва}{}{
{\begin{itemize}[leftmargin=0.6cm, label={\textbullet}]
\item Коммерческая разработка сервисов РЛС для последующей интеграцией в МО.
\item Разработка сервиса "РЛС-Противопоказания" основанного на анализе противопоказаний из инструкций лекарственных препаратов.
\item Автоматизация разметки, используя Pandas и регулярные выражения, для последующего использование NLTK.
\item Обучение лингвистической модели на основе BERT для задачи NER.
\item Прототипа сервиса. Разработка API на Django, нативный фронт с использованием CSS фреймворков. Адаптирование и предоставление данных из PostgreSQL. 
\\
\end{itemize}}}

\customcventry{09/2023}{}{Хакатон INNOGLOBALHACK}{Иннополис, Казань}{}{
{\begin{itemize}[leftmargin=0.6cm, label={\textbullet}]
\item Интеграция API сервиса для организации спринтов в Telegram.
\item Разработка телеграм бота на aiogram для работы с CRUD методами API сервиса.
\\
\end{itemize}}}


\customcventry{10/2023}{}{Инженерный хакатон Сеченова}{Москва}{}{
{\begin{itemize}[leftmargin=0.6cm, label={\textbullet}]
\item Разработка мобильного приложения с интегрированием CNN модели для распознования и классификации эмоций людей с расстройством аутистического спектра.
\item Организация разметки изображений с помощью DropBox.
\item Обучение CNN модели с помощью Keras, TensorFlow, тестирование на OpenCV.
\item Конвертация модели в TensorFlowLight и её интеграция в локальную архитектуру мобильного приложения на Flutter.
\end{itemize}}}